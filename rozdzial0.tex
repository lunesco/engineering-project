\section*{Streszczenie pracy}

Celem niniejszej pracy było opracowanie oraz stworzenie aplikacji, która kategoryzuje treści tekstowe dynamicznie ładowane na stronach internetowych. Klient podaje na wejście naszego programu adres URL, a jako wynik otrzymuje podaną przez siebie stronę (w HTML-u), która zawiera kolorowe ramki na zaklasyfikowanych treściach, każda innej barwy. Moduł, który przyjmuje wejście użytkownika został zaimplementowany w Javie, natomiast model językowy, który klasyfikuje treści, został napisany w języku Python z pomocą biblioteki Flair.

Praca ma charakter naukowy, gdyż jej głównym celem było wytrenowanie modelu językowego. Tematyka ta jest związana z popularnym ostatnio uczeniem maszynowym, a także przetwarzaniem języka naturalnego. Zarówno cała dziedzina uczenia maszynowego, jak również metody przetwarzania języków naturalnych bardzo szybko ewoluują, a nowe narzędzia powstają stosunkowo często. Dlatego też możliwość ich zastosowania w określonym problemie praktycznym napotyka na wiele wyzwań.

\section*{Abstract of engineer's thesis}

The purpose of this work was to develop and create an application that categorizes text content dynamically loaded on websites. The customer gives the URL to our program as the input, and as a result he receives the page he provided (in HTML), which contains colored frames on the classified content, each of different hue. The module that accepts user input was implemented in Java, while the language model that classifies content was written in Python using the Flair library.

The work is scientific because its main purpose was to train the language model. This subject is related to the recently popular machine learning and natural language processing. Both the entire field of machine learning as well as natural language processing methods evolve very quickly and new tools are created relatively often. Therefore, the possibility of their application in a particular practical problem faces many challenges.
















